\documentclass[prodmode,acmtecs]{acmsmall} % Aptara syntax

% % Package to generate and customize Algorithm as per ACM style
% \usepackage[ruled]{algorithm2e}
% \renewcommand{\algorithmcfname}{ALGORITHM}
% \SetAlFnt{\small}
% \SetAlCapFnt{\small}
% \SetAlCapNameFnt{\small}
% \SetAlCapHSkip{0pt}
% \IncMargin{-\parindent}

% Metadata Information
%\acmVolume{9}
%\acmNumber{4}
%\acmArticle{39}
%\acmYear{2010}
%\acmMonth{3}



\usepackage[utf8]{inputenc} % Caracteres con acentos.
\usepackage{color,xcolor,colortbl}


\usepackage{listings}
\lstdefinelanguage{Erlang}%
  {morekeywords={abs,after,and,apply,atom,atom_to_list,band,binary,%
      binary_to_list,binary_to_term,bor,bsl,bsr,bxor,case,catch,%
      date,div,element,erase,begin,end,exit,export,float,float_to_list,%
      halt,hash,hd,if,info,import,integer,integer_to_list,%
      fun,link,list_to_atom,list_to_float,list_to_integer,%
      list_to_tuple,module,nodes,now,of,or,port,ports,%
      processes,receive,reference,register,registered,rem,%
      round,self,setelement,size,spawn,throw,time,tl,trace,trunc,%
      tuple,tuple_to_list,unlink,unregister,whereis,try,%
      infinity,undefined,when},%
   otherkeywords={->,!,[,],\{,\}},%
   morecomment=[l]\%,%
   morestring=[b]",%
   morestring=[b]'%
  }

\lstset{language=Erlang}
\lstset{basicstyle=\small\ttfamily}

\renewcommand{\lstlistingname}{Code}

%% ======================
%% My variables -- just change things here!
\newcommand{\UPM}{Universidad Politécnica de Madrid, Spain}
\newcommand{\DepUPM}{Babel Group, DLSIIS, Facultad de Informática}
\newcommand{\mailRJ}{rjrodriguez@fi.upm.es}
\newcommand{\mailTo}[1]{\url{#1}}

\newcommand{\correspondingText}{Corresponding author: Ricardo~J~Rodríguez.
Campus de Montegancedo, Facultad de Informática
{Dpto. de Lenguajes y Sistemas Inform\'{a}ticos e Ingenier\'{i}a de Software},
Universidad Politécnica de Madrid. 28660 Boadilla del Monte (Madrid), Spain.
Email: {\mailTo{\mailRJ}}. Phone: (+34) 913365017 Fax: (+34) 913363669.}

\newcommand{\paperTitle}{TBD}
\newcommand{\paperTitleShort}{\paperTitle}
\newcommand{\authorNames}{Ricardo J. Rodríguez, Lars-\r{A}ke Fredlund and Ángel
Herranz}
\newcommand{\authorNamesShort}{R.J. Rodríguez, L. Fredlund, A. Herranz}
\newcommand{\authorsMailTo}{\{rjrodriguez, lfredlund, aherranz\}@fi.upm.es}
\newcommand{\keywordsList}{UML, Erlang, automatic code generation}

\newcommand{\ackText}{This work was partially supported by ARTEMIS Joint
Undertaking nSafeCer under grant agreement no. 295373 and from National
funding.}
%% ======================

\newcommand{\IGNORE}[1]{}
\newcommand{\REVIEW}[1]{\textcolor{black}{#1}}
\newcommand{\TODO}[1]{\textcolor{red}{#1}}

\newcommand{\mcErlang}{{\tt McErlang}}



\keywords{\keywordsList} % Keywords for the article

\usepackage{hyperref}
\hypersetup{
  baseurl={\mailRJ},
  pdftitle={{\paperTitle} | {\authorNamesShort}},%
  pdfauthor={\authorNames},%
  pdfproducer={\LaTeX},%
  pdfsubject={{\paperTitle} | {\authorNamesShort}},%
  pdfkeywords={\keywordsList}%
}

\renewcommand*{\doi}[1]{}
\sloppy

% Document starts
\begin{document}

% Page heads
\markboth{R.J. Rodríguez et al.}{\paperTitle -- Submitted to Special Issue on
Application of Concurrency to System Design}

% Title portion
\title{\paperTitle}
\author{RICARDO J. RODRÍGUEZ
\affil{Technical University of Madrid, Spain}
LARS-\r{A}KE FREDLUND
\affil{Technical University of Madrid, Spain}
ÁNGEL HERRANZ
\affil{Technical University of Madrid, Spain}}

\begin{abstract}
There is a growing interest in the use of UML for the specification and
analysis of the requirements of critical systems.
%
A key factor for the successful adoption of UML and model driven
methodologies is the possibility to validate designs in early stages
so as to correct possible mistakes or even reveal inconsistent or
incomplete requirements in their specifications.
%
One approach to achieve this goal is to translate subsets of UML into
programs that can be processed by automatic verification tools
such as model checkers.  However, the subsets of UML considered are
typically very limited and no formal definition of the translation is provided.
%
In this paper we define a translation of UML models containing
class, object and state diagrams into Erlang programs that can be verified with
model checking.
%
The translation is applied to an industrial case study on the
modelling and verification of the requirements of a high-speed railway
signalling system. We show how errors in the model and ambiguities in
the requirements are automatically detected.
\end{abstract}

%TODO
%\category{C.2.2}{Computer-Communication Networks}{Network Protocols}

%TODO
\terms{TODO}

\keywords{\keywordsList}

\acmformat{{\authorNames}, 2013. {\paperTitle}.}
% At a minimum you need to supply the author names, year and a title.
% IMPORTANT:
% Full first names whenever they are known, surname last, followed by a period.
% In the case of two authors, 'and' is placed between them.
% In the case of three or more authors, the serial comma is used, that is, all
% author names except the last one but including the penultimate author's name
% are followed by a comma,
% and then 'and' is placed before the final author's name.
% If only first and middle initials are known, then each initial
% is followed by a period and they are separated by a space.
% The remaining information (journal title, volume, article number, date, etc.)
% is 'auto-generated'.

\begin{bottomstuff}
\ackText

Author's addresses: R.~J. Rodríguez, L-A. Fredlund {and} A. Herranz,
Campus de Montegancedo, Facultad de Informática
{Dpto. de Lenguajes y Sistemas Inform\'{a}ticos e Ingenier\'{i}a de Software},
Technical University of Madrid (Spain). Email: {\authorsMailTo}.
\end{bottomstuff}

\maketitle


\TODO{\ldots}

\newpage

\bibliographystyle{ACM-Reference-Format-Journals}
%\bibliography{}

\end{document}

%%% Local Variables: 
%%% mode: latex
%%% TeX-master: t
%%% TeX-PDF-mode: t
%%% ispell-local-dictionary: "british"
%%% End: 
