\section{UML Subset}
\label{sec:umlsubset}

\todo{Develop these bullet points}
\begin{itemize}
\item UML es ambiguo, e impreciso.
\item Mostrar ejemplos de diagramas inconsistentes, tanto estáticos
  como dinámicos.
\item A nivel de ejemplos de máquina de estado podemos incluso hablar
  del scope de símbolos accesibles para justificar nuestra relación
  entre máquinas y clases.
\item Poner ejemplos de máquinas que no se sabe bien qué significan.
\item Seleccionar y justificar el subset (está en correos electrónicos).
\item Explicar la semántica a través de los pequeños ejemplos
  mostrados anteriormente.
\end{itemize}

\todo{Texto traído del paper de UML a Uppaal}
The UML models considered in the formalism are divided into the static
model (to represent class diagrams and object diagrams), and the
dynamic model (to represent state diagrams). Describing and
integrating the intended meaning of class diagrams and object diagrams
is not straightforward but it is not extremely difficult. Although we
do not contemplate all the constructions of both diagrams, most of
those that we leave out are considered \emph{syntactic sugar} (can be
expressed in function of other constructions). Omitting the rest of
constructions is justified because they were unnecessary in our domain
and/or because of our search of an economical mathematical
representation.

Our restrictions to obtain a subset of the language of state diagrams
are clearly harder than the restrictions on the static model. In this
case, our justification has to do, mainly, with the simplicity of the
semantics. Just to mention one example, the allowed state diagrams are
\emph{behavioral state machines} to specify the behaviour of the
instances of a given class that are based on the Harel's visual
formalism of statecharts (\cite{harel}).


%%% Local Variables: 
%%% mode: latex
%%% TeX-PDF-mode: t
%%% ispell-local-dictionary: "british"
%%% End: 
